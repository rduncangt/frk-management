\documentclass[10pt]{article}
\usepackage[
    paperwidth=8.5in,
    paperheight=11in,
    left=0.25in,
    right=0.25in,
    top=0.25in,
    bottom=0.25in,
    footskip=0pt,
]{geometry}
\usepackage[most]{tcolorbox}
\usepackage{xcolor}
\usepackage{tabularx}
\usepackage{calc}

% Custom tcolorbox style for bin labels
\newtcbox{\binbox}[1][]{
  on line,
  arc=5pt,
  boxsep=0pt,
  left=3pt,
  right=3pt,
  top=2pt,
  bottom=2pt,
  colback=#1!10!white,
  colframe=#1!50!white,
  boxrule=0.5pt,
  nobeforeafter,
  tcbox raise base,
}

% Define colors for different categories
\definecolor{Abin}{HTML}{A8E0D9}
\definecolor{Bbin}{HTML}{D5D5FF}
\definecolor{Cbin}{HTML}{FF0000}
\definecolor{Dbin}{HTML}{FCBE85}
\definecolor{Obin}{gray}{0.9}

% Define a macro for the bin content
\newcommand{\bincontent}[2]{%
    % The \parbox must subtract the left and right padding of the tcolorbox (3pt + 3pt = 6pt)
    \parbox{\dimexpr \linewidth - 6pt\relax}{%
        \centering
        \vspace*{2.0mm} % Space at the top
        \textbf{#1}\\[1mm]
        \small #2
        \vspace*{2.0mm} % Space at the bottom
    }%
}

%---------------------------------------------------------------
% lengths for the layout
%---------------------------------------------------------------
\ExplSyntaxOn
% https://tex.stackexchange.com/a/123283/5764
\DeclareExpandableDocumentCommand { \printlengthas } { m m }
  { \dim_to_decimal_in_unit:nn {#1} { 1 #2 } #2 }
\ExplSyntaxOff 


\setlength{\tabcolsep}{0pt}

\begin{document}

% lengths calculated based on \linewidth must be after \begin{document}
\newlength{\boxwidth} % width for the outer box
\newlength{\awidth} % width for A bins
\newlength{\aheight} % height for A bins
\newlength{\bwidth} % width for B bins
\newlength{\bheight} % height for B bins
\newlength{\threetimesbwidth} % width for the open area (3x B bin width)
\newlength{\dshort} % width for B bins
\newlength{\dlong} % height for B bins

% \setlength{\boxwidth}{\linewidth}
\setlength{\boxwidth}{\dimexpr \linewidth - 2\arrayrulewidth\relax}
\setlength{\awidth}{\dimexpr \boxwidth/5 - 1\arrayrulewidth\relax} % 9 cm
\setlength{\aheight}{5\awidth/9}% 5 cm
\setlength{\bwidth}{\awidth} % 9 cm
\setlength{\bheight}{11\bwidth/9}% 11 cm
\setlength{\threetimesbwidth}{\dimexpr 3\bwidth + 2\arrayrulewidth\relax}
\setlength{\dlong}{\dimexpr \boxwidth/4 - 1\arrayrulewidth\relax} % 11 cm
\setlength{\dshort}{8\dlong/11} % 7.5 cm

\typeout{---------------------------------------------------} % Use typeout to print the lengths for debugging

\pagestyle{empty}

% \begin{center}
%     \Large\bfseries FRK Tool Box Layout
%     \vspace{5mm}
% \end{center}

% husky tool box
% dimensions from measuring the actual box
% A bins: 9cm wide x 5cm high
% B bins: 9cm wide x 11cm high
% D bins: 11cm wide x 7.5cm high

%== Top Tray Layout ==========================================
%
%   ---------------------------------------------------
%   |    A1   |    A2   |    A3   |    A4   |    A5   |
%   ---------------------------------------------------
%   |         |                             |         |
%   |    B1   |              O1             |    B2   |
%   |         |                             |         |
%   ---------------------------------------------------
%   |    A6   |    A7   |    A8   |    A9   |    AA   |
%   ---------------------------------------------------
%
% alternative layout:
%
%   ---------------------------------------------------
%   |         |    A1   |    A2   |    A3   |    A4   |
%   |    B1   |---------------------------------------|
%   |         |                             |         |
%   |---------|              O1             |    B2   |
%   |         |                             |         |
%   |    B3   |---------------------------------------|
%   |         |    A5   |    A6   |    A7   |    A8   |
%   ---------------------------------------------------

%=============================================================
\begin{center}
  \begin{tabular}{|p{\boxwidth}|}
    \hline
    % Row 1: Five A bins -----------------------------------------------------------------------------
    \begin{tabularx}{\boxwidth}{|p{\awidth}|p{\awidth}|p{\awidth}|p{\awidth}|p{\awidth}|}
      \colorbox{Abin}{%
      \parbox[c][\aheight][c]{\awidth - 2\fboxsep}%
        {
          {\footnotesize \textbf{A1: unassigned} \\
          % 1. part 1
          % 2. part 2
          }%
        }%
      } &
      \colorbox{Abin}{%
      \parbox[c][\aheight][c]{\awidth - 2\fboxsep}%
        {
          {\footnotesize \textbf{A2: needle cages} \\
          % 1. part 1
          % 2. part 2
          }%
        }%
      } &
      \colorbox{Abin}{%
      \parbox[c][\aheight][c]{\awidth - 2\fboxsep}%
        {
          {\footnotesize \textbf{A3: sprockets} \\
          % 1. part 1
          % 2. part 2
          }%
        }%
      } &
      \colorbox{Abin}{%
      \parbox[c][\aheight][c]{\awidth - 2\fboxsep}%
        {
          {\footnotesize \textbf{A4: e-clips and washers} \\
          % 1. part 1
          % 2. part 2
          }%
        }%
      } &
      \colorbox{Abin}{%
      \parbox[c][\aheight][c]{\awidth - 2\fboxsep}%
        {
          {\footnotesize \textbf{A5: decompression valves} \\
          % 1. part 1
          % 2. part 2
          }%
        }%
      }%
      \\
    \end{tabularx}%
    \\
    \hline

    % Row 2: Two B bins and open area ----------------------------------------------------------------
    \begin{tabularx}{\boxwidth}{|p{\bwidth}|p{\threetimesbwidth}|p{\bwidth}|}
      \colorbox{Bbin}{%
        \parbox[c][\bheight][c]{\dimexpr\bwidth - 2\fboxsep\relax}%
        {
          {\footnotesize \textbf{B1: chain brake} \\
          % 1. Tension spring	3	261	1141 160 5500	chain brake
          % 2. part 2
          }%
        }%
      } &
      % open area in the middle
      \colorbox{Obin}{%
      \parbox[c][\bheight][c]{\dimexpr\threetimesbwidth - 2\fboxsep\relax}{\centering\textbf{OPEN AREA} \\air filters\\filler caps}%
      } &
      \colorbox{Bbin}{%
        \parbox[c][\bheight][c]{\dimexpr\bwidth - 2\fboxsep\relax}%
        {
          {\footnotesize \textbf{B2: fuel filters} \\
          % 1. part 1
          % 2. part 2
          }%
        }%
      }                                                                                                 \\
    \end{tabularx}%
    \\

    % Row 3: Five A bins -----------------------------------------------------------------------------
    \begin{tabularx}{\boxwidth}{|p{\awidth}|p{\awidth}|p{\awidth}|p{\awidth}|p{\awidth}|}
      \hline
      \colorbox{Abin}{%
      \parbox[c][\aheight][c]{\awidth - 2\fboxsep}%
        {
          {\footnotesize \textbf{A6: guards} \\
          % 1. part 1
          % 2. part 2
          }%
        }%
      } &
      \colorbox{Abin}{%
      \parbox[c][\aheight][c]{\awidth - 2\fboxsep}%
        {
          {\footnotesize \textbf{A7: bumper strips} \\
          % 1. part 1
          % 2. part 2
          }%
        }%
      } &
      \colorbox{Abin}{%
      \parbox[c][\aheight][c]{\awidth - 2\fboxsep}%
        {
          {\footnotesize \textbf{A8: chain catchers} \\
          % 1. part 1
          % 2. part 2
          }%
        }%
      } &
      \colorbox{Abin}{%
      \parbox[c][\aheight][c]{\awidth - 2\fboxsep}%
        {
          {\footnotesize \textbf{A9: screws and nuts} \\
          % 1. part 1
          % 2. part 2
          }%
        }%
      } &
      \colorbox{Abin}{%
      \parbox[c][\aheight][c]{\awidth - 2\fboxsep}%
        {
          {\footnotesize \textbf{AA: pawls and springs} \\
          % 1. part 1
          % 2. part 2
          }%
        }%
      }%
      \\
      \hline
    \end{tabularx}%
    \\
    \hline
  \end{tabular}
\end{center}

\vfill
\noindent
cut here \hfill cut here
\hrule
\noindent
\raisebox{\depth}{\scalebox{1}[-1]{cut here}}
\hfill \raisebox{\depth}{\scalebox{1}[-1]{cut here}}
\vfill

%== Bottom Section Layout ====================================
%
%   ---------------------------------------------------
%   |         |                             |         |
%   |    D1   |                             |    D2   |
%   |         |                             |         |
%   |---------+                             +---------|
%   |         |                                       |
%   |    D3   |              O2                       |
%   |         |                                       |
%   |---------+---------+                             |
%   |         |         |                             |
%   |    D4   |    D5   |                             |
%   |         |         |                             |
%   ---------------------------------------------------
%
%=============================================================
\begin{center}
  \begin{tabular}{|p{\boxwidth}|}
    \hline
    % Row 1 ------------------------------------------------------------------------------------------
    \begin{tabularx}{\boxwidth}{|p{\dlong}|p{2\dlong}|p{\dlong}|}
      % \hline
      \colorbox{Dbin}{%
      \parbox[c][\dshort][c]{\dlong - 2\fboxsep}
        {
          {\footnotesize \textbf{D1: tensioner hardware} \\
          % 1. part 1
          % 2. part 2
          }%
        }%
      } &
      % open area in the middle
      \colorbox{Obin}{%
        \parbox[c][\dshort][c]{\dimexpr 2\dlong - 2\fboxsep\relax}{\,}%
      } &
      \colorbox{Dbin}{%
        \parbox[c][\dshort][c]{\dlong - 2\fboxsep}%
        {
          {\footnotesize \textbf{D2: clutches} \\
          % 1. part 1
          % 2. part 2
          }%
        }%
      }%
      \\
    \end{tabularx}%
    \\

    % Row 2 ------------------------------------------------------------------------------------------
    \begin{tabularx}{\boxwidth}{|p{\dlong}|p{3\dlong}|}
      \colorbox{Dbin}{%
      \parbox[c][\dshort][c]{\dlong - 2\fboxsep}%
        {
          {\footnotesize \textbf{D3: oil pump hardware} \\
          % 1. part 1
          % 2. part 2
          }%
        }%
      } &
      % open area in the middle
      \colorbox{Obin}{%
        \parbox[c][\dshort][c]{\dimexpr\boxwidth - 1\dlong - 2\fboxsep\relax}{\centering\textbf{OPEN AREA}%
      \\rewind starter assemblies%
      \\brake bands}%
      }%
      \\
    \end{tabularx}%
    \\

    % Row 3 ------------------------------------------------------------------------------------------
    \begin{tabularx}{\boxwidth}{|p{\dlong}|p{\dlong}|p{2\dlong}|}
      \colorbox{Dbin}{%
      \parbox[c][\dshort][c]{\dlong - 2\fboxsep}%
        {
          {\footnotesize \textbf{D4: spark plugs} \\
          % 1. part 1
          % 2. part 2
          }%
        }%
      } &
      \colorbox{Dbin}{%
      \parbox[c][\dshort][c]{\dlong - 2\fboxsep}%
        {
          {\footnotesize \textbf{D5: rim sprocket kits} \\
          % 1. part 1
          % 2. part 2
          }%
        }%
      } &
      % open area in the middle
      \colorbox{Obin}{%
        \parbox[c][\dshort][c]{\dimexpr\boxwidth - 2\dlong - 2\fboxsep\relax}{\,}%
      }%
      \\
      \hline
    \end{tabularx}%
  \end{tabular}
\end{center}

\end{document}