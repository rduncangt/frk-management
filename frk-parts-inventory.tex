\documentclass[10pt]{article}
\usepackage{geometry}
\usepackage{booktabs}
\usepackage[l3]{csvsimple}
\usepackage{tabularx}
\usepackage{array}
\usepackage{fancyhdr}

\geometry{letterpaper, top=0.45in, bottom=0.15in, left=0.25in, right=0.25in}

\setlength{\headheight}{14pt}
\pagestyle{fancy}
\lhead{\footnotesize FRK Inventory}
\rhead{\footnotesize\today} 
\renewcommand{\headrulewidth}{0.3pt}
\renewcommand{\footrulewidth}{0pt}

\begin{document}
\pagestyle{fancy}

\begin{tabularx}{\textwidth}{
    >{\hsize=0.2\hsize}X% bin 
    >{\hsize=0.6\hsize\raggedleft\arraybackslash}X% count
    >{\hsize=2.0\hsize}X% partname 
    >{\hsize=1.0\hsize}X% sku 
    >{\hsize=0.6\hsize}X% saws 
    >{\hsize=1.6\hsize}X% category
    }
    \toprule
    \textbf{Bin} & \textbf{Qty/Req} & \textbf{Part Name} & \textbf{SKU} \ & \textbf{Saws} & \textbf{Category} \\
    \midrule
    
    % The core command:
    % 1. Reads 'data.tsv'
    % 2. Sets the delimiter to 'tab'
    % 3. Skips the first line (header)
    % 4. Defines temporary variables for the needed columns
    % 5. For each row, prints the content of the variables separated by '&' and ends with '\\'
    \csvreader[
        separator=tab,
        no head,
        before line=\\[0.01cm], % ensures the line ends with a new row command
        late after line=\\[0.1cm]\hline, % ensures the line ends with a new row command
        late after last line={}
    ]{frk_items.tsv}{1=\sku, 2=\partname, 3=\count,4=\saws, 6=\bin, 7=\category}
    {
        \bin & \count & \partname & \sku & \saws & \category
    }

    % \bottomrule
\end{tabularx}

\end{document}
